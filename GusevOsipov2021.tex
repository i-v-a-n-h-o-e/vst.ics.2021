% todo [] ORCID и должность МИ
% todo [] Заголовок x4
% todo [] Финансирование x2
% todo [] Аннотация x2
% todo [] Ключевые слова
% todo [] Введение
% todo [] Основные понятия и определения Система, МД и уравнение беллмана
% todo [] Линейный случай
% todo [] Нелинейный случай
% todo [] Билинейная система
% todo [] Осцилятор
% todo [] Уницикл
% todo [] Список литературы x2


%%% Стилевой файл Вестника. Не изменять!
\documentclass[12pt,a4paper,twoside]{article}  %MikTeX
\usepackage[utf8]{inputenc}%Win
\usepackage[T1,T2A]{fontenc}
\usepackage{amsmath,amsfonts,amssymb,amscd,euscript}
\usepackage[english, russian]{babel}

\usepackage{graphicx}
\usepackage{psfrag}
\usepackage{cite}
\usepackage{caption}
\captionsetup[figure]{labelfont={bf},labelformat={default},labelsep=period,name={Рис.}}
\captionsetup[table]{labelfont={bf},labelformat={default},labelsep=period,name={Таблица}}

\usepackage[textwidth=166mm,top=2cm,textheight=250mm,left=20mm,showframe=false]{geometry}
\tolerance=500
\voffset=0.5cm
%\headsep=5mm
%\textwidth=166mm
%\textheight=250mm
%\oddsidemargin=-4mm
%\evensidemargin=-4mm
%\topmargin=-7mm

\usepackage{ifthen}
\newcommand{\serieseng}{\ifthenelse{\equal{МАТЕМАТИКА}{\seriesrus}}{MATHEMATICS}{\ifthenelse{\equal{МЕХАНИКА}{\seriesrus}}{MECHANICS}{\ifthenelse{\equal{КОМПЬЮТЕРНЫЕ НАУКИ}{\seriesrus}}{COMPUTER SCIENCE}{UNDEFINED}}}}

\makeatletter

\global\let\@fundingrus\@empty
\def\fundingrus#1{\gdef\@fundingrus{\@ifempty{#1}{\@empty}{\noindent{\bf Финансирование.} #1}}}

\global\let\@fundingeng\@empty
\def\fundingeng#1{\gdef\@fundingeng{\@ifempty{#1}{\@empty}{\noindent{\bf Funding.} #1}}}

%информация о статье на русском языке
\def\titlerus{\begin{otherlanguage}{russian}\thispagestyle{firstpagestylerus}\label{paperfirstpage}
		\hbox{УДК \UDC} \vspace{30pt plus 6pt}
		\begin{flushleft}
			{\bf\copyright~{\textit{\authorsrus}}\\[2ex]
				{\MakeUppercase{\articletitlerus}}}
		\end{flushleft}
	\end{otherlanguage}%
}

\def\annotationandkeywordsrus{\begin{otherlanguage}{russian}
		\noindent{\small \referatrus \par}
		\vspace{8pt}
		\noindent {\small {\it Ключевые слова}: \keywordsrus} \par%
		\vspace{8pt}
		\noindent {\small {\textrm DOI:} \paperdoi} \par%
		\vspace{10pt plus 6pt minus 1pt}
\end{otherlanguage}}


%информация о статье на английском языке
\def\titleeng{\clearpage
	\thispagestyle{firstpagestyleeng}
	\begin{otherlanguage}{english}\noindent\parbox{\textwidth}{\small\noindent\textbf{\textit{\authorseng}}}\par
		\vspace{2pt plus 1pt minus 0pt}\par
		\noindent\parbox{\textwidth}{\small\noindent\textbf{\articletitleeng}}\par
		\vspace{6pt plus 1pt minus 0pt}\par
		\noindent\parbox{\textwidth}{\small\noindent{\it Keywords}: \keywordseng} \par
		\vspace{8pt}
		\noindent\small{MSC2020: }\MSC\par%
		\vspace{7pt} \par
		\noindent {\small {\textrm DOI:} \paperdoi} \par
\end{otherlanguage}}

\def\refereng{\begin{otherlanguage}{english}\vspace{20pt}  \par \small  \noindent \referateng\par\vspace{3ex}\@fundingeng\par\end{otherlanguage}}

%Поступила в редакцию
\def\receivedrus{\begin{otherlanguage}{russian}\vspace{3ex} \hfill Поступила  в редакцию~~\datereceive \par \vspace{5ex} \par\end{otherlanguage}}

\def\receivedeng{\begin{otherlanguage}{english}\vspace{3ex} \hfill Received~~\datereceive \par \vspace{5ex} \par\end{otherlanguage}}

%контактная информация об авторах
\def\contactsrus{\begin{otherlanguage}{russian}\noindent\parbox{\textwidth}{\small\noindent\contactinformationrus\par\vspace{20pt}\par
			\noindent{\bf Цитирование: }\authorsrus.~\articletitlerus~// Вестник Удмуртского университета. Математика. Механика. Компьютерные науки. \paperyear. Т.~\papervolume. Вып.~\papernumber. \mbox{С.~\pageref{paperfirstpage}--\pageref{paperlastpage}.}}\end{otherlanguage}}

\def\contactseng{\begin{otherlanguage}{english}\noindent\parbox{\textwidth}{\small\noindent\contactinformationeng\par\vspace{10pt}\par
			\noindent{\bf Citation: }\authorseng. \articletitleeng, {\it Vestnik Udmurtskogo Universiteta. Matematika. Mekhanika. Komp'yuternye Nauki}, \paperyear, vol.~\papervolume, issue~\papernumber, \mbox{pp.~\pageref{paperfirstpage}--\pageref{paperlastpage}.}}\label{paperlastpage}\end{otherlanguage}}


\@addtoreset{equation}{section}
\renewcommand{\section}{\@startsection{section}{1}{0pt}{1.3ex
		plus 1ex minus .1ex}{1.3ex plus .1ex}{\bf\,\S\,}}
\newcommand{\point}{\hspace*{-4mm}{\bf.}\;}
\newcommand{\sect}[1]{\begin{flushleft}%
		\protect{\section{\point#1}}\end{flushleft}}

\renewcommand{\@begintheorem}[2]{\begin{trivlist}
		\item[\hspace{\labelsep}{\bf \mbox{~~~}#1\ #2.}]}
	\renewcommand{\@opargbegintheorem}[3]{\begin{trivlist}
			\item[\hspace{\labelsep}{\bf \mbox{~~~}#1\ #2 {\rm (#3).}}]}
		\renewcommand{\@endtheorem}{\end{trivlist}}
	
	\newtheorem{teo}{Теорема}
	\newtheorem{lem}{Лемма}
	\newtheorem{pre}{Предложение}
	\newtheorem{utv}{Утверждение}
	\newtheorem{sle}{Следствие}
	\newtheorem{hyp}{Гипотеза}
	\newtheorem{df}{Определение}
	\newtheorem{zam}{Замечание}
	\newtheorem{pr}{Пример}
	\newtheorem{assum}{Предположение}
	\newtheorem{cond}{Условие}
	
	\newcommand{\doc}{\mbox{Д о к а з а т е л ь с т в о}}
	
	
	\renewcommand{\@evenfoot}{}
	\renewcommand{\@oddfoot}{}
	
	\let\OLDthebibliography\thebibliography
	\renewcommand\thebibliography[1]{
		\OLDthebibliography{#1}
		\setlength{\parskip}{0pt}
		\setlength{\itemsep}{0pt plus 0.3ex}
	}
	\renewcommand*{\@biblabel}[1]{#1.\hfill}
	
	
	\newcommand{\sectnn}[1]{%
		\begin{center}%
			\large\section*{#1}%
		\end{center}%
	}
	
	\renewcommand{\@evenhead}%
	{%
		\begin{otherlanguage}{russian}%
			\raisebox{0pt}%
			[\headheight]%
			[0pt]%
			{%
				\vbox{\hbox to\textwidth{\thepage\strut\hfil\authorsrus\hfil}\hrule\vspace{8pt}
			}}%
		\end{otherlanguage}%	
	}
	
	\renewcommand{\@oddhead}%
	{%
		\begin{otherlanguage}{russian}%
			\raisebox{0pt}%
			[\headheight]%
			[0pt]%
			{%
				\vbox{\hbox to\textwidth{\strut\hfil\articleshorttitlerus\hfil\thepage}\hrule\vspace{8pt}% \hbox to \textwidth{\series \hfil  \issue}
	}}\end{otherlanguage}}
	
	\makeatother
	
	\usepackage{fancyhdr}
	
	\fancypagestyle{firstpagestylerus}{
		\fancyhf{}
		\headheight=30pt
		\fancyhead[C]{\parbox{\textwidth}{{\scriptsize ВЕСТНИК\hfill УДМУРТСКОГО\hfill УНИВЕРСИТЕТА.\hfill МАТЕМАТИКА.\hfill МЕХАНИКА.\hfill КОМПЬЮТЕРНЫЕ\hfill НАУКИ}
				\vspace{2pt}
				\hrule
				\vspace{8pt}
				{\small \hbox to \textwidth{\seriesrus \hfill  \paperyear. Т.~\papervolume. Вып.~\papernumber. С.~\pageref{paperfirstpage}--\pageref{paperlastpage}.}}}}
		\renewcommand{\headrulewidth}{0.0pt}
	}
	
	\fancypagestyle{basestylerus}{
		\fancyhf{}
		\headheight=15pt
		\fancyhead[LE,RO]{\thepage}
		\fancyhead[CE]{\articleshorttitlerus}
		\fancyhead[CO]{\authorsrus}
		\renewcommand{\headrulewidth}{0.4pt}
	}
	
	\fancypagestyle{firstpagestyleeng}{
		\fancyhf{}
		\headheight=30pt
		\fancyhead[C]{\parbox{\textwidth}{{\scriptsize VESTNIK \hfill UDMURTSKOGO \hfill UNIVERSITETA. \hfill MATEMATIKA. \hfill MEKHANIKA. \hfill KOMP'UTERNYE \hfill NAUKI}\vspace{2pt}
				\hrule
				\vspace{8pt}
				{\small \hbox to \textwidth{\serieseng \hfill  \paperyear. Vol.~\papervolume. Issue~\papernumber. Pp.~\pageref{paperfirstpage}--\pageref{paperlastpage}.}}}}
		\renewcommand{\headrulewidth}{0pt}
	}
	
	\fancypagestyle{basestyleeng}{
		\fancyhf{}
		\headheight=15pt
		\fancyhead[LE,RO]{\thepage}
		\fancyhead[CE]{\articleshorttitleeng}
		\fancyhead[CO]{\authorseng}
		\renewcommand{\headrulewidth}{0.4pt}
	}
	
	\pagestyle{basestylerus} 

%%% Стилевой файл Вестника. Не изменять!!!

\usepackage{refcheck}
%\usepackage[times]{pscyr}
%%% Подраздел Серии. МАТЕМАТИКА или МЕХАНИКА или КОМПЬЮТЕРНЫЕ НАУКИ
\newcommand{\seriesrus}{МАТЕМАТИКА}

\newcommand{\paperyear}{2021} %%% Год
\newcommand{\papervolume}{31} %%% Том
\newcommand{\papernumber}{1}  %%% Выпуск

%%% Фамилии автора (авторов). Не забываем пробелы!
\newcommand{\authorsrus}{М.\,И.~Гусева, И.\,О.~Осипов} %%% на русском языке (точку в конце не ставим)
\newcommand{\authorseng}{M.\,I.~Gusev, I.\,O.~Osipov} %%% на английском языке (точку в конце не ставим)

%%% Полное название статьи на русском языке! -
%%% (точку в конце не ставим)
\newcommand{\articletitlerus}{Об одном методе локального синтеза для нелинейных систем с интегральными ограничениями}
%%% Сокращенное название статьи на русском языке! -
%%% первые несколько слов, многоточие не ставим! (точку в конце не ставим)
\newcommand{\articleshorttitlerus}{К вопросу о маршрутизации перемещений}
%%% Полное название статьи на английском языке! -
%%% (точку в конце не ставим)
\newcommand{\articletitleeng}{On the question of the optimization of permutations in the problem with dynamic constraints}
%%% Сокращенное название статьи на английском языке! -
%%% первые несколько слов, многоточие не ставим! (точку в конце не ставим)
\newcommand{\articleshorttitleeng}{On the question of the optimization of permutations}

%%% Классификаторы. УДК. Mathematical Subject Classification (не более 3).
\newcommand{\UDC}{517.977.1} %%% Проставляет автор!!!
\newcommand{\MSC}{93B03} %%% Проставляет автор!!!

\newcommand{\paperdoi}{10.35634/vmXXXXXX}%здесь будет номер DOI, ведущий на doi.org в электронном варианте

\fundingrus{Исследования первого автора выполнены при финансовой поддержке Министерства науки и высшего образования РФ в рамках базовой части госзадания в сфере науки, проект №\,1.1234.2017/8.9. Исследования второго автора выполнены при финансовой поддержке РФФИ в рамках научного проекта 18--01--01234.} %Ссылка на грант на русском языке

\fundingeng{The study of the first author was funded by the Ministry of Science and Higher Education of the Russian Federation in the framework of the basic part, project no. 1.1234.2017/8.9. The study of the second author was funded by RFBR, project number 18--01--01234.} %Ссылка на грант на английском языке

%%% Аннотация статьи на русском. Представляет собой полноценный реферат статьи.
%%% Нельзя использовать ссылки на литературу в аннотации.
%%% Допустимы математические формулы в ``чистом'' latex-e (без сокращений).
\newcommand{\referatrus}{В данной статье рассматривается задача построения локального синтеза для нелинейной управляемой системы с интегральными ограничениями на управление. Рассматривается задача о  приведении движения управляемой системы в начало координат при заданном ресурсе управления. При этом считается, что промежуток времени, в течении которого осуществляется перевод системы, достаточно мал. Указаны классы нелинейных систем, для которых задачу можно решить путем приближенной замены нелинейной системы линеаризованной.}
%%% Аннотация статьи на английском. Представляет собой полноценный реферат статьи.
%%% Рекомендуемый объем 100-250 слов.
%%% Допустимы математические формулы в ``чистом'' latex-e (без сокращений).
\newcommand{\referateng}{We consider }
%%% Ключевые слова на русском и английском языках (не более 10 слов).
\newcommand{\keywordsrus}{линейные системы с последействием, приводимость, показатели Ляпунова, ляпуновские инварианты.}
\newcommand{\keywordseng}{linear systems with delay, reducibility, Lyapunov exponents, Lyapunov invariants.}

%%% Дата поступления (можно ставить дату отправки) статьи.
\newcommand{\datereceive}{01.12.2021} %%% формат дд.мм.гггг

%%% В контактной информации об авторе указывается:
%%% Фамилия, Имя, Отчество (обязательно полностью), ученая степень (по желанию),
%%% ученое звание (по желанию), должность (обязательно), место работы (обязательно)
%%% не обязательно подробно (кафедра, факультет, отдел), достаточно лишь название организации,
%%% адрес организации (обязательно) - индекс, страна, город, улица, дом. \\ С новой строки ORCID \\ С новой строки E-mail
%%% Не нужно указывать домашний адрес, нужно указывать рабочий адрес.
\newcommand{\contactinformationrus}{Гусев Михаил Иванович,
д.\,ф.-м.\,н., ведущий научный сотрудник, отдел оптимального управления,
Институт математики и механики УрО РАН, 620219, Россия,  г. Екатеринбург, ул.
С.~Ковалевской, 16.\\
ORCID: https://orcid.org/0000-0000-0000-XXXX \\
E-mail: gmi@imm.uran.ru
Осипов Иван Олегович,
аспирант, отдел оптимального управления,
Институт математики и механики УрО РАН, 620219, Россия,  г. Екатеринбург, ул.
С.~Ковалевской, 16.\\
ORCID: https://orcid.org/0000-0003-3071-535X \\
E-mail: i.o.osipov@imm.uran.ru
}
%%% Информация об авторе на английском языке:
%%% Фамилия, Имя, Отчество (обязательно полностью), Ученая Степень (по желанию),
%%% Ученое Звание (по желанию), Должность (обязательно), Место Работы (обязательно)
%%% не обязательно подробно, достаточно лишь название организации (используется официальное
%%% название организации на английском языке; если официального названия организации на английском языке
%%% не существует, то название пишется транслитом),
%%% адрес организации (обязательно) в следующем формате (согласно House Style Guide, стр. 53) -
%%% улица, дом, город, индекс, страна. С новой строки ORCID. С новой строки E-mail
\newcommand{\contactinformationeng}{Gusev Michail Ivanovich, Professor, Department of Optimal Control,
Institute of Mathematics and Mechanics, Ural Branch of the Russian Academy of Sciences, ul. S. Kovalevskoi, 16, Yekaterinburg, 620219,  Russia. \\
ORCID: https://orcid.org/0000-0000-0000-XXXX \\
E-mail: gmi@imm.uran.ru \\ [5pt]
Osipov Ivan Olegovich, Post-Graduate student, Department of Optimal Control,
Institute of Mathematics and Mechanics, Ural Branch of the Russian Academy of Sciences, ul. S. Kovalevskoi, 16, Yekaterinburg, 620219,  Russia. \\
ORCID: https://orcid.org/0000-0003-3071-535X \\
E-mail: i.o.osipov@imm.uran.ru
}
%---------------------------------------
%%% Следующая команда устанавливает двойную нумерацию формул. Если двойная
%%% нумерация не нужна, то эту команду следует закомментировать.
\renewcommand{\theequation}{\arabic{section}.\arabic{equation}}
%%% Если Вы закомментировали предыдущую команду, то не пользуйтесь командой \sect
%%% нумерующей параграфы статьи. Если при этом Вам необходимо обозначать разделы
%%% статьи, то используйте окружение, как для Введения: \noindent{\bf{\S\,1. Название раздела}}
%%% при этом параграфы нумеруются вручную
%---------------------------------------
%%% Здесь могут быть "свои" макрокоманды, например так
\newcommand{\pX}{{\mathcal X}}
\let\msf=\mathsf
\newcommand{\var}{\mathop{\sf Var}}
\newcommand*\rfrac[2]{{}^{#1}\!/_{#2}}
%%% однако их количество не должно быть большим.
%%% здесь нельзя вставлять сокращения, которые не используются.
%%% можно пользоваться возможностями подключенных пакетов
%%% {amsmath,amsfonts,amssymb,amscd,euscript}, которые находятся в файле vum.tex
%---------------------------------------
\setcounter{page}{1} % Номер страницы начала статьи (проставляет ответственный за выпуск)
%---------------------------------------
\begin{document}

\titlerus

\annotationandkeywordsrus

%----------------------------------
\begin{flushleft}
{\bf{Введение}}
\end{flushleft}
%%% Введение не нумеруется (поэтому команда \sect не применяется)

% с красной строки



%--------------------------------------
\sect{Основные обозначения и определения}
%%% обратите внимание на оформление длинного тире после доллара
Рассмотрим нелинейную систему, аффинную по управлению
\begin{gather}\label{nonlinear}
	\begin{gathered}
		\dot{x}(t)=f(x(t))+B u(t), \qquad t_0 \leqslant t \leqslant t_0 + \bar{\varepsilon}, \qquad x(t_0) = x_0, \\
	\end{gathered}
\end{gather}
Здесь $ x \in \mathbb{R}^n $ -- вектор состояния, $ u \in \mathbb{R}^r $ -- управление,  $ \bar{\varepsilon} $ --- некоторое фиксированное положительное число.


Под $ \mathbb{L}_2 = \mathbb{L}_2[t_0,t_0+\bar{\varepsilon}]  $ будем понимать,   пространство интегрируемых с квадратом скалярных или вектор-функций  на $ [t_0,t_0+\bar{\varepsilon}] $.  Скалярное произведение в $ \mathbb{L}_2 $ определено равенством
\begin{gather*}
	\left(u(\cdot),\upsilon(\cdot) \right) = \int_{t_0}^{t_0+\bar{\varepsilon}}u^{\top}(t)\upsilon(t) \, dt.
\end{gather*}
 Управление $ u(\cdot) $ будем выбирать из шара радиуса $ \mu, \mu > 0 $
\begin{gather}\label{constr}
	\lVert u(\cdot)\rVert^2_{\mathbb{L}_2} = \left(u(\cdot),u(\cdot) \right) \leqslant \mu^2
\end{gather}
в пространстве $\mathbb{L}_2[t_0,t_0+\bar{\varepsilon}]$ вектор-функций.
В условиях описанных предположений, каждому $ u(\cdot) \in \mathbb{L}_2 $ соответствует единственное абсолютно непрерывное решение $ x(t)=x(t,x_0, u(\cdot)) $ системы \eqref{nonlinear}, определённое на интервале $ [t_0,t_0+\bar{\varepsilon}] $.

Все траектории $ x(t) $ системы \eqref{nonlinear}, отвечающие удовлетворяющим \eqref{constr} управлениям,  лежат внутри некоторого компактного множества $ D \subset \mathbb{R}^n $.

%Дадим здесь несколько определений, которыми будем пользоваться в дальнейшем.
Пусть $ 0 <  \varepsilon \leqslant \bar{\varepsilon} $.
\begin{df}
	{\it Множеством нуль управляемости} $ N(\varepsilon) $ системы \eqref{nonlinear} в пространстве состояний в момент времени $ t_0 + \varepsilon $ назовем
	множество всех начальных состояний $ x(t_0) \in \mathbb{R}^n $ системы \eqref{nonlinear},  из которых система может быть приведена в начало координат управлениями
	$ u(t) \in B_{\mathbb{L}_2}(0,\mu)  =\left\lbrace u:\lVert u(\cdot)\rVert^2_{\mathbb{L}_2} \leqslant \mu^2\right\rbrace  $,
	\begin{gather*}
		N(\varepsilon)=\{\widetilde{x}\in \mathbb{R}^n:\exists u(\cdot)\in B_{\mathbb{L}_2}(0,\mu),\; x(t_0 + \varepsilon,\widetilde{x},u(\cdot)) = 0\}.
	\end{gather*}
\end{df}

В приведённом определении можно считать, что $ \mathbb{L}_2 =\mathbb{L}_2[t_0,t_0+\varepsilon] $, либо  $ \mathbb{L}_2 =\\=\mathbb{L}_2[t_0,t_0+\bar{\varepsilon}] $. Нетрудно понять, что для любого из этих пространств мы получаем одно и то же множество нуль-управляемости. Будем далее считать, что $ \mathbb{L}_2 =\mathbb{L}_2[t_0,t_0+\varepsilon] $.

C другой стороны, если на пространстве состояний системы \eqref{nonlinear} и временном интервале $ t_0 \leqslant t \leqslant t_0 + \varepsilon $ ввести функцию $ V(t,x) $ минимального ресурса, необходимого для приведения системы из начального состояния $x$ в начало координат
\begin{gather}\label{Bellman_fun}
	V(t,x) = \min\limits_{u} \int_{t_0}^{t} u^{\top}(t) u(t) \, dt.
\end{gather}
то множество нуль-управляемости $ N(\varepsilon) $ может быть найдено, как множество уровня функции Беллмана
\begin{gather}\label{Null_set}
	N(\varepsilon)  = \{x \in \mathbb{R}^n: V(t_0 + \varepsilon,x) \leqslant \mu^2\}
\end{gather}
Функция $V(t,x)$  удовлетворяет уравнению
\begin{gather}\label{Bellman_eq}
	\frac{\partial V(t,x)}{\partial t} = -\min\limits_{u} \{u^{\top}(t) u(t) + \left(\frac{\partial V(t,x)}{\partial x}\right)^{\top} \left(f(x(t))+B u(t)\right) \}.
\end{gather}
Закон обратной связи, на котором достигается минимум в уравнении \eqref{Bellman_eq}
\begin{gather}\label{feedback}
	u(t,x) = -\frac{1}{2} f_2^{\top}(t,x(t))\frac{\partial V(t,x)}{\partial x}.
\end{gather}

%---------------------------------
\sect{Линейный случай}
%%% если Вы отказались от двойной нумерации (это Ваше право), то окружение
%%% \sect использовать нельзя. В этом случае для выделения разделов статьи
%%% используйте окружение, как для введения:
%%% \begin{flushleft}
%%% {\bf{\S\,2. Инвариантные и вполне регулярные множества}}
%%% \end{flushleft}
%%% Нет резона использовать двойную нумерацию, если количество нумерованных
%%% формул и теорем небольшое.

%%% На все нумерованные формулы должны быть ссылки.
%%% Не следует нумеровать формулы, на которые нет ссылок.
%%% Проверяется подключением пакета \usepackage{refcheck}

%%% ключевые слова в определениях можно (нужно) выделить курсивом
%%% (\sf не применяется, только \it)
Рассмотрим здесь линейную систему
\begin{gather}\label{linear}
	\dot{x} =  A  x + B u, \qquad t_0 \leqslant t \leqslant t_0 + \varepsilon, \qquad x(t_0) = x_0,
\end{gather}

Пусть $ X(\tau_1,\tau_0) = \Phi(\tau_1) \Phi(\tau_0) $, где $\Phi(t) $ -- матрица Коши, удовлетворяющая уравнению 
\begin{gather*}
	\dot{\Phi}(t) = A \Phi(t), \qquad \Phi(t_0) = I
\end{gather*}
Решение системы \eqref{linear} в момент времени $ t$, $t_0 \leqslant t \leqslant t_0 + \varepsilon $ имеет форму
\begin{gather*}
	x(t) = X(t,t_0) x(t_0) + \int_{t_0}^{t} X(t,\tau) B u(\tau)  d\tau.
\end{gather*}
Приравнивая $x(t) $ к нулю, выражаем $x(t_0) $
\begin{gather*}
	x(t_0) = -\int_{t_0}^{t} X(t_0,\tau) B u(\tau) d\tau.
\end{gather*} 

 Теперь найдем опорную функцию $ \rho(l|N(\varepsilon))$, для этого возьмем произвольный вектор  $l \in \mathbb{R}^n$, $ l \neq 0 $ и найдем максимум скалярного произведения $ (l,x(t_0)) $ на всех $x(t_0) \in N(\varepsilon)$:
 \begin{gather}\label{sup_fun}
 	\begin{gathered}
 			\rho(l|N(\varepsilon)) = \max\limits_{x(t_0) \in N(\varepsilon)} (l,x(t_0))  = \sup\limits_{u \in B_{\mathbb{L}_2}(0,\mu)} -\int_{t_0}^{t_0+\varepsilon} l^{\top}X(t_0,\tau) B u(\tau) d\tau =\\=  \sup\limits_{u \in B_{\mathbb{L}_2}(0,\mu)} (w_l(\cdot),u(\cdot)) = \mu \| w_l(\cdot)\| = \mu \sqrt{l^{\top} W(t_0,t_0+\varepsilon) l},
 	\end{gathered}
 \end{gather}
где $ w_l(t) = -B^{\top}X^{\top}(t_0,t) l$, а симметричная матрица $ W(t_0,t_0+\varepsilon) $ определяется равенством
\begin{gather}\label{W}
	W(t_0,t_0+\varepsilon) = \int_{t_0}^{t_0+\varepsilon} X(t_0,\tau) B B^{\top} X^{\top} (t_0,\tau) d\tau.
\end{gather}
\eqref{sup_fun} -- опорная функция эллипсоиода, то есть $ N(\varepsilon) = \{x: x^{\top} W^{-1}(t_0,t_0+\varepsilon)x \leqslant \mu^2\}$. С другой стороны, множество нуль управляемости --- множество уровня функции Беллмана \eqref{Null_set}. Таким образом, в случае линейной системы функция Беллмана \eqref{Bellman_fun} имеет вид
\begin{gather}\label{Bellman_linear_fun}
	V(t,x) = x^{\top} Q(t) x,
\end{gather}
где матрица  $ Q(t) $ может быть найдена из уравнения 
\begin{gather}\label{eqQ}
	\dot{Q}  = Q B B^{\top} Q - A^{\top}Q - Q A,
\end{gather}
c учетом $ W^{-1}(t_0,t) = Q(t_0) $.
Уравнение \eqref{eqQ} получено подстановкой \eqref{Bellman_linear_fun} в \eqref{Bellman_eq} с учетом того, что оптимальная обратная связь \eqref{feedback} может быть записана в виде
\begin{gather}\label{linear_feedback}
	u(t,x) = -B^{\top} Q(t) x
\end{gather}

Алгоритм построения обратной связи приводящей систему \eqref{linear} в начало координат выглядит следующим образом:
\begin{description}
	\item[Шаг 1.] Рассчитываем матрицу $ W(t_0,t_0+\varepsilon) $ по формуле \eqref{W}.
	\item[Шаг 2.] Интегрируем уравнение \eqref{eqQ} с начальным условием $ Q(t_0)  = W^{-1}(t_0,t_0+\varepsilon)$.
	\item[Шаг 3.] Рассчитываем управление $ u(t,x) = -B^{\top} Q(t) x(t)$
\end{description}
Замкнутую обратной связью \eqref{linear_feedback} систему \eqref{linear} преобразуем к виду
\begin{gather}\label{feedback_linear_system}
	\left\lbrace \begin{array}{l}
			\dot{x} = (A - B B^{\top} Q(t) ) x, \qquad t_0 \leqslant t \leqslant t_0 + \varepsilon, \qquad x(t_0) = x_0\\
			\dot{Q} = Q B B^{\top} Q - A^{\top}Q - Q A, \qquad Q(t_0) = W^{-1}(t_0,t_0+\varepsilon).
		\end{array} \right. 
	\end{gather}
\begin{teo}\label{linear_teo}
	Траектория $x(t) $ системы \eqref{feedback_linear_system} выпущенная из точки $ x $ попадает в начало координат в момент времени $ t_0 + \varepsilon$. Расход интегрального ресурса управления на переход из $ x $ в $ 0 $ равен $x^{\top} Q(t_0) x $ -- это минимально возможное значение ресурса.
\end{teo}
\doc. 
	Продифференцируем $x^{\top} Q x$ вдоль траектории $ x(t) $:
	\begin{gather*}
		\frac{d}{dt} x^{\top} Q x = \dot{x}^{\top} Q x + x^{\top} \dot{Q} x + x^{\top} Q \dot{x} = x^{\top} (A^{\top} - Q B B^{\top} )Q x + \\ + x^{\top} Q (A - B B^{\top} Q)x + x^{\top} (Q B B^{\top} Q - A^{\top}Q - Q A) x = \\ = 
		-x^{\top} (Q B B^{\top} Q) x.
	\end{gather*}
	Интегрируя последнее равенство от $ t_0 $ до $ t_0 + \varepsilon $ получаем
	\begin{gather}\label{xqx}
		x^{\top}(t) Q(t)x(t) = x^{\top}(t_0) Q(t_0)x(t_0) - \int_{t_0}^{t_0 + \varepsilon} x^{\top}(\tau) Q(\tau) B B^{\top} Q(\tau) x(\tau) d\tau= \\ = 
		x^{\top}(t_0) Q(t_0)x(t_0) - \int_{t_0}^{t_0 + \varepsilon} u^{\top}(\xi)  u(\xi) d\xi,
	\end{gather}
где $ u(\xi) = -B^{\top} Q(\xi) x(\xi)$ -- текущее управление.

Рассмотрим множество $ Y(t) = \left\lbrace x:x^{\top} Q(t) x  \leqslant c^2 \right\rbrace $, где $ c $ -- заданная константа. Опорная функция множества $ Y(t) $ равна 
\begin{gather*}
	\rho(l|Y(t))  = c \left( l^{\top} Q^{-1}(t) l\right)^{\rfrac{1}{2}} =  c \left( l^{\top} W(t_0,2t_0 + \varepsilon - t) l\right)^{\rfrac{1}{2}} 
\end{gather*}
Из определения $ W $  следует, что существует $ k > 0 $, такое что $ \left\| W(t_0,2t_0 + \varepsilon - t)\right\| \leqslant k (\varepsilon - t) $, поэтому \begin{gather*}
	\rho(l|Y(t)) \leqslant c\sqrt{k} \sqrt{\varepsilon - t} \| l\| .
\end{gather*}
Из последнего неравенства следует, что для любого $ x \in Y(t) $ справедлива оценка 
\begin{gather*}
	 \| x\| \leqslant  c\sqrt{k} \sqrt{\varepsilon - t} 
 \end{gather*}
Учитывая, что $ x^{\top}(t) Q(t)x(t) \leqslant  x^{\top}(t_0) Q(t_0)x(t_0)$, мы можем заключить, что 
\begin{gather*}
	 \| x\| \leqslant  \left(k (\varepsilon - t) x^{\top}(t_0) Q(t_0)x(t_0)\right)^{\rfrac{1}{2}}
\end{gather*}

Таким образом, $ x(t) \rightarrow 0 $ при $ t \rightarrow t_0 + \varepsilon $. Покажем, что $ 	x^{\top}(t) Q(t)x(t) \rightarrow 0 $ при $ t \rightarrow t_0 + \varepsilon $. Этот факт не следует из стремления к 0 $ x(t) $, так как $ Q(t) = W^{-1}(t_0,2t_0+\varepsilon - t) $ не ограничена вблизи $ t_0 + \varepsilon $ в силу $ W(t_0,t_0) = 0 $. Из неравенства 
\begin{gather}
	\frac{d}{dt}x^{\top} Q x \leqslant 0, x^{\top} Q x > 0
\end{gather}
следует, что $ x^{\top}(t) Q(t)x(t) $ не возрастает и ограничена снизу. Таким образом, существует предел 
\begin{gather}
	\lim\limits{t \rightarrow t_0 + \varepsilon} = q \geqslant 0.
\end{gather}
Покажем, что $ q = 0 $. Допустим противное. Переходя в \eqref{xqx} к пределу при $ t \rightarrow t_0 + \varepsilon $ получим
\begin{gather}
	\int_{t_0}^{t_0+\varepsilon} u^{\top}(\xi)  u(\xi) d\xi = x^{\top} Q(t_0) x - q.
\end{gather}
\hfill $\square$
%------------------------------
\sect{Нелинейный случай}



Вероятно, описанный подход к синтезу управления может быть распространен и на более общий случай систем вида
\begin{gather*}
		\dot{x}(t)=f_1(t,x(t))+f_2(t,x(t))u(t).
\end{gather*}
Иллюстрирующий данный факт пример приведен в следующем разделе
%------------------------------
\sect{Примеры}


%%% используем \widehat вместо \hat
%%% используем \widetilde вместо \tilde
%%% используем \overline вместо \bar
%%% для многоточия используем \ldots


%----------------------------------------------
\vspace{1ex}

\makeatletter
\@fundingrus\par
\@addtoreset{equation}{section}
\@addtoreset{footnote}{section}
\renewcommand{\section}{\@startsection{section}{1}{0pt}{1.3ex
plus 1ex minus 1ex}{1.3ex plus .1ex}{}}

\vspace{3ex}
\small
{ %\scriptsize

\renewcommand{\refname}{{\rm\centerline{СПИСОК ЛИТЕРАТУРЫ}}}

\begin{thebibliography}{99}

%%%%  Не допускаются ссылки на еще не опубликованные статьи
%%%%
%%%%  ОФОРМЛЕНИЕ ЛИТЕРАТУРЫ (ВНИМАТЕЛЬНО СМОТРИМ НА ЗНАКИ ПРЕПИНАНИЯ
%%%%  И ПОРЯДОК СЛЕДОВАНИЯ ВЫХОДНЫХ ДАННЫХ)
%%%%  ОФОРМЛЕНИЕ ПРИВЕДЕНО В СООТВЕТСТВИИ С ГОСТом Р 7.0.5-2008
%%%%
%%%% Для всех статей (и других источников, при наличии) следует указывать DOI следующим образом (в виде ссылки)
%%%% https://doi.org/10.20537/vm130302
%%%% Само слово DOI не пишем
%%%% Если DOI отсутствует, следует указывать url статьи на сайте журнала, или на %%%% агрегаторах: Mathnet, Elibrary, zbmath, Elsevier, SpringerLink и др.
%%%% Ссылки долны быть проверяемы.

%%%% Если статья опубликована на русском языке, то в англоязычном списке литературы в конце выходных данных статьи нужно написать (in Russian). (NEW с 01.01.2015)
%%%% Между различными областями библиографического описания тире не ставим (NEW с 01.01.2012)
%%%% Написание пробелов между инициалами и фамилией  упростилось (NEW с 01.01.2012)
%%%% Нигде не пишутся знаки ~ (тильда) \, (backslash с запятой) (если будет необходимо, их проставит редактор)
%%%% Между инициалами пробела нет, между фамилией и инициалами один пробел
%%%% Следите за пробелами после слов: том (Т. 1), номер (№ 1), выпуск (Вып. 1), страницы (С. 1--3)!
%%%% Следите за диапазонами страниц (двойное тире)! С. 20--30
%%%% Различные элементы выходных данных разделяются точками: Год. Том. Номер. Страницы
%%%% 1991. Т. 1. № 1. С. 10--20. После символа № точка не ставится.
%%%%	
%%%% Монография:
%%%% Фамилия И.О. Название книги. Город: Издательство, Год. Страницы.
\bibitem{Abgar} Абгарян К.А. Матричное исчисление с приложениями в теории динамических систем. М.: Физматлит, 1994. 544 с.

%%%% Учебник или учебное пособие:
%%%% Фамилия И.О. Название учебника. Номер издания (факультативно).
%%%% Город: Издательство, Год. Страницы.


%%%% Ссылка на диссертацию
%%%% (Указывается организация, в которой защищена диссертация.)


%%%% Ссылка на автореферат диссертации
%%%% (В выходных данных указывается город, в котором защищена диссертация,
%%%% а не место печатания реферата. Наименование организации необязательно.)


%%%%  Статья в журнале в центральном или зарубежном издании
%%%%  Фамилия И.О. Название статьи // Название журнала. Год. Том. Номер.
%%%%  Страницы.


%%%%  Статья в журнале в центральном или зарубежном издании
%%%%  Фамилия И.О. Название статьи // Название журнала. Год. Том. Выпуск.
%%%%  Страницы.
%%%%  Необходимо указывать doi статьи
%%%%  В конце адреса точка не ставится


%%%%  Статья в журнале Известия Института математики и информатики Удмуртского государственного университета. Укзываем DOI, если есть.


%%%%  Статья в журнале Известия Института математики и информатики Удмуртского государственного университета. Если DOI нет, указываем URL  на Mathnet


%%%% Депонированная статья

%%%% Тезисы докладов конференции



%%%% Статья в сборнике


%%%%ссылки на литературу из интернета- пишем адрес


%%%% Статья в журнале Труды ИММ УрО РАН


% указывается doi русскоязычной версии статьи


% статья в сборнике статей (книге) под редакцией



% Conference paper


\end{thebibliography}}

\receivedrus

\contactsrus

\titleeng

\pagestyle{basestyleeng}

\refereng




%%%%  Не допускаются ссылки на еще не опубликованные статьи
%%%%
%%%%  СПИСОК ЛИТЕРАТУРЫ ЛАТИНИЦЕЙ (ВНИМАТЕЛЬНО СМОТРИМ НА ЗНАКИ ПРЕПИНАНИЯ
%%%%  И ПОРЯДОК СЛЕДОВАНИЯ ВЫХОДНЫХ ДАННЫХ)
%%%%  ОТЛИЧАЕТСЯ ОТ РОССИЙСКОГО ГОСТА
%%%%
%%%% Для всех статей (и других источников, при наличии) следует указывать DOI следующим образом (в виде ссылки)
%%%% https://doi.org/10.20537/vm130302
%%%% Само слово DOI не пишем
%%%% Если DOI отсутствует, следует указываем url статьи на сайте журнала, или на агрегаторах: Mathnet, Elibrary, zbmath, Elsevier, SpringerLink и др.
%%%% Ссылки долны быть проверяемы.
%%%%
%%%% Монография:
%%%% Фамилия И.О. Название книги транслитом (курсивом) (Перевод названия на английский),
%%%% Город: Издательство, Год, страницы.
\selectlanguage{english}
\begin{center}
REFERENCES
\end{center}
\setlength{\leftmargini}{1.8em}
\begin{enumerate}
\setlength{\itemsep}{0em}
\parskip=0pt



%%%% Если иностранная книга переведена на русский язык, то оформляется так
%%%%


%%%% Указываются все авторы монографии, слово and перед последним автором не ставится


% Книга, переведенная с русского на английский

%%%% Ссылка на диссертацию
%%%% (Указывается город)


%%%% Ссылка на автореферат диссертации
%%%% (В выходных данных указывается город, в котором защищена диссертация,
%%%% а не место печатания реферата. Наименование организации необязательно.)


%%%%  Статья в журнале в центральном или зарубежном издании
%%%%  Фамилия И.О. Название статьи на английском языке,
%%%%  Название журнала транслитом (курсивом), Год, том, номер, страницы (пишутся две буквы p, если страниц больше чем одна)
%%%%  Если существует перевод статьи на английский язык, то указываются выходные данные переводной статьи. Если статья имеет DOI, его необходимо указать.


%%%%  Статья в журнале Вестник Удмуртского университета. Математика. Механика. Компьютерные науки
%%%%  Фамилия И.О. Название статьи на английском языке, Название журнала транслитом (курсивом), Год, номер, страницы.


%%%%  Статья в журнале Известия Института математики и информатики Удмуртского государственного университета. Укзываем DOI, если есть.



%%%%  Статья в журнале Известия Института математики и информатики Удмуртского государственного университета. Если DOI нет, указываем URL (eng) на Mathnet


%%%% Депонированная статья

%%%% Тезисы докладов конференции



%%%% Статья в сборнике



%%%% ссылки на литературу из интернета- пишем адрес



%%%% Статья в журнале Труды ИММ УрО РАН (если ссылка дается на оригинальную статью)


%%%% Статья в журнале Труды ИММ УрО РАН (если ссылка дается на статью в переводной версии журнала)

% указывается doi англоязычной версии статьи



% статья в сборнике статей (книге) под редакцией


% Conference paper
\item

\end{enumerate}

\receivedeng

\contactseng


\end{document} 