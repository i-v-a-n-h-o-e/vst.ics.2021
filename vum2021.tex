%%% Стилевой файл Вестника. Не изменять!
\documentclass[12pt,a4paper,twoside]{article}  %MikTeX
\usepackage[utf8]{inputenc}%Win
\usepackage[T1,T2A]{fontenc}
\usepackage{amsmath,amsfonts,amssymb,amscd,euscript}
\usepackage[english, russian]{babel}

\usepackage{graphicx}
\usepackage{psfrag}
\usepackage{cite}
\usepackage{caption}
\captionsetup[figure]{labelfont={bf},labelformat={default},labelsep=period,name={Рис.}}
\captionsetup[table]{labelfont={bf},labelformat={default},labelsep=period,name={Таблица}}

\usepackage[textwidth=166mm,top=2cm,textheight=250mm,left=20mm,showframe=false]{geometry}
\tolerance=500
\voffset=0.5cm
%\headsep=5mm
%\textwidth=166mm
%\textheight=250mm
%\oddsidemargin=-4mm
%\evensidemargin=-4mm
%\topmargin=-7mm

\usepackage{ifthen}
\newcommand{\serieseng}{\ifthenelse{\equal{МАТЕМАТИКА}{\seriesrus}}{MATHEMATICS}{\ifthenelse{\equal{МЕХАНИКА}{\seriesrus}}{MECHANICS}{\ifthenelse{\equal{КОМПЬЮТЕРНЫЕ НАУКИ}{\seriesrus}}{COMPUTER SCIENCE}{UNDEFINED}}}}

\makeatletter

\global\let\@fundingrus\@empty
\def\fundingrus#1{\gdef\@fundingrus{\@ifempty{#1}{\@empty}{\noindent{\bf Финансирование.} #1}}}

\global\let\@fundingeng\@empty
\def\fundingeng#1{\gdef\@fundingeng{\@ifempty{#1}{\@empty}{\noindent{\bf Funding.} #1}}}

%информация о статье на русском языке
\def\titlerus{\begin{otherlanguage}{russian}\thispagestyle{firstpagestylerus}\label{paperfirstpage}
\hbox{УДК \UDC} \vspace{30pt plus 6pt}
\begin{flushleft}
{\bf\copyright~{\textit{\authorsrus}}\\[2ex]
{\MakeUppercase{\articletitlerus}}}
\end{flushleft}
\end{otherlanguage}%
}

\def\annotationandkeywordsrus{\begin{otherlanguage}{russian}
\noindent{\small \referatrus \par}
\vspace{8pt}
\noindent {\small {\it Ключевые слова}: \keywordsrus} \par%
\vspace{8pt}
\noindent {\small {\textrm DOI:} \paperdoi} \par%
\vspace{10pt plus 6pt minus 1pt}
\end{otherlanguage}}


%информация о статье на английском языке
\def\titleeng{\clearpage
\thispagestyle{firstpagestyleeng}
\begin{otherlanguage}{english}\noindent\parbox{\textwidth}{\small\noindent\textbf{\textit{\authorseng}}}\par
\vspace{2pt plus 1pt minus 0pt}\par
\noindent\parbox{\textwidth}{\small\noindent\textbf{\articletitleeng}}\par
\vspace{6pt plus 1pt minus 0pt}\par
\noindent\parbox{\textwidth}{\small\noindent{\it Keywords}: \keywordseng} \par
\vspace{8pt}
\noindent\small{MSC2020: }\MSC\par%
\vspace{7pt} \par
\noindent {\small {\textrm DOI:} \paperdoi} \par
\end{otherlanguage}}

\def\refereng{\begin{otherlanguage}{english}\vspace{20pt}  \par \small  \noindent \referateng\par\vspace{3ex}\@fundingeng\par\end{otherlanguage}}

%Поступила в редакцию
\def\receivedrus{\begin{otherlanguage}{russian}\vspace{3ex} \hfill Поступила  в редакцию~~\datereceive \par \vspace{5ex} \par\end{otherlanguage}}

\def\receivedeng{\begin{otherlanguage}{english}\vspace{3ex} \hfill Received~~\datereceive \par \vspace{5ex} \par\end{otherlanguage}}

%контактная информация об авторах
\def\contactsrus{\begin{otherlanguage}{russian}\noindent\parbox{\textwidth}{\small\noindent\contactinformationrus\par\vspace{20pt}\par
\noindent{\bf Цитирование: }\authorsrus.~\articletitlerus~// Вестник Удмуртского университета. Математика. Механика. Компьютерные науки. \paperyear. Т.~\papervolume. Вып.~\papernumber. \mbox{С.~\pageref{paperfirstpage}--\pageref{paperlastpage}.}}\end{otherlanguage}}

\def\contactseng{\begin{otherlanguage}{english}\noindent\parbox{\textwidth}{\small\noindent\contactinformationeng\par\vspace{10pt}\par
\noindent{\bf Citation: }\authorseng. \articletitleeng, {\it Vestnik Udmurtskogo Universiteta. Matematika. Mekhanika. Komp'yuternye Nauki}, \paperyear, vol.~\papervolume, issue~\papernumber, \mbox{pp.~\pageref{paperfirstpage}--\pageref{paperlastpage}.}}\label{paperlastpage}\end{otherlanguage}}


\@addtoreset{equation}{section}
\renewcommand{\section}{\@startsection{section}{1}{0pt}{1.3ex
plus 1ex minus .1ex}{1.3ex plus .1ex}{\bf\,\S\,}}
\newcommand{\point}{\hspace*{-4mm}{\bf.}\;}
\newcommand{\sect}[1]{\begin{flushleft}%
\protect{\section{\point#1}}\end{flushleft}}

\renewcommand{\@begintheorem}[2]{\begin{trivlist}
\item[\hspace{\labelsep}{\bf \mbox{~~~}#1\ #2.}]}
\renewcommand{\@opargbegintheorem}[3]{\begin{trivlist}
\item[\hspace{\labelsep}{\bf \mbox{~~~}#1\ #2 {\rm (#3).}}]}
\renewcommand{\@endtheorem}{\end{trivlist}}

\newtheorem{teo}{Теорема}
\newtheorem{lem}{Лемма}
\newtheorem{pre}{Предложение}
\newtheorem{utv}{Утверждение}
\newtheorem{sle}{Следствие}
\newtheorem{hyp}{Гипотеза}
\newtheorem{df}{Определение}
\newtheorem{zam}{Замечание}
\newtheorem{pr}{Пример}
\newtheorem{assum}{Предположение}
\newtheorem{cond}{Условие}

\newcommand{\doc}{\mbox{Д о к а з а т е л ь с т в о}}


\renewcommand{\@evenfoot}{}
\renewcommand{\@oddfoot}{}

\let\OLDthebibliography\thebibliography
\renewcommand\thebibliography[1]{
  \OLDthebibliography{#1}
  \setlength{\parskip}{0pt}
  \setlength{\itemsep}{0pt plus 0.3ex}
}
\renewcommand*{\@biblabel}[1]{#1.\hfill}


\newcommand{\sectnn}[1]{%
\begin{center}%
\large\section*{#1}%
\end{center}%
}

\renewcommand{\@evenhead}%
{%
\begin{otherlanguage}{russian}%
\raisebox{0pt}%
[\headheight]%
[0pt]%
{%
\vbox{\hbox to\textwidth{\thepage\strut\hfil\authorsrus\hfil}\hrule\vspace{8pt}
}}%
\end{otherlanguage}%	
}

\renewcommand{\@oddhead}%
{%
\begin{otherlanguage}{russian}%
\raisebox{0pt}%
[\headheight]%
[0pt]%
{%
\vbox{\hbox to\textwidth{\strut\hfil\articleshorttitlerus\hfil\thepage}\hrule\vspace{8pt}% \hbox to \textwidth{\series \hfil  \issue}
}}\end{otherlanguage}}

\makeatother

\usepackage{fancyhdr}

\fancypagestyle{firstpagestylerus}{
\fancyhf{}
\headheight=30pt
\fancyhead[C]{\parbox{\textwidth}{{\scriptsize ВЕСТНИК\hfill УДМУРТСКОГО\hfill УНИВЕРСИТЕТА.\hfill МАТЕМАТИКА.\hfill МЕХАНИКА.\hfill КОМПЬЮТЕРНЫЕ\hfill НАУКИ}
\vspace{2pt}
\hrule
\vspace{8pt}
{\small \hbox to \textwidth{\seriesrus \hfill  \paperyear. Т.~\papervolume. Вып.~\papernumber. С.~\pageref{paperfirstpage}--\pageref{paperlastpage}.}}}}
\renewcommand{\headrulewidth}{0.0pt}
}

\fancypagestyle{basestylerus}{
\fancyhf{}
\headheight=15pt
\fancyhead[LE,RO]{\thepage}
\fancyhead[CE]{\articleshorttitlerus}
\fancyhead[CO]{\authorsrus}
\renewcommand{\headrulewidth}{0.4pt}
}

\fancypagestyle{firstpagestyleeng}{
\fancyhf{}
\headheight=30pt
\fancyhead[C]{\parbox{\textwidth}{{\scriptsize VESTNIK \hfill UDMURTSKOGO \hfill UNIVERSITETA. \hfill MATEMATIKA. \hfill MEKHANIKA. \hfill KOMP'UTERNYE \hfill NAUKI}\vspace{2pt}
\hrule
\vspace{8pt}
{\small \hbox to \textwidth{\serieseng \hfill  \paperyear. Vol.~\papervolume. Issue~\papernumber. Pp.~\pageref{paperfirstpage}--\pageref{paperlastpage}.}}}}
\renewcommand{\headrulewidth}{0pt}
}

\fancypagestyle{basestyleeng}{
\fancyhf{}
\headheight=15pt
\fancyhead[LE,RO]{\thepage}
\fancyhead[CE]{\articleshorttitleeng}
\fancyhead[CO]{\authorseng}
\renewcommand{\headrulewidth}{0.4pt}
}

\pagestyle{basestylerus} 